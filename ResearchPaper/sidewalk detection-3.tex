\documentclass[a4class]{article}
\usepackage{dhucs}
\title{Sidewalk detection based on line detection}
\author{Dongmin Kim}
\begin{document}
\maketitle
\begin{abstract}
In this paper, we present the development of an computer vision algorithm which can detect sidewalks based on line detection. The main functions of algorithm are inducing pedestrians to follow right ways to across sidewalks. Most sidewalk detection algorithms are using color histogram analysis to extract the area of sidewalk. However, our algorithm is based on the line detection because there are various kinds of patterns and colors used in detecting sidewalks all over the world. The development of the algorithms is performed using a Walking Assistance Mobile Application for visually impaired people and can be used in no extra device and technology.
\end{abstract}
\section{Introduction}
As the technology has been renovated each day, more convenient life for ours are also requried. In our sociey, there are some obstacles that visually impaired person have to be suffered in living in high densed urban. Walking in road, as a pedestrian, it's so hard to them to go straight with no visual guide to helpe recognizing sidewalk's circumstance. To solve this problem, there are some sidewalk detection method, one by Seng(Seng, John S., and Thomas J. Norrie. "Sidewalk following using color histograms." Journal of Computing Sciences in Colleges 23.6 (2008): 172-180.) and there are many paper about sidewalk detection. But they are extracting data based on color models of histogram and analsysis that to get information. Thus, in the situation of color models won't work properly, such as existing some obstacles at sidewalk or image's color is not fixed, it's reliablity has dameged. So, we determined to develop sidewalk detection algorithms based on line detection method and set two main focus of the algorithm. 
\newline First is that the algorithm is need to be based on line detection. To cope every situation that occured in sidewalk such as fallen leaves, unarranged sidewalk blocks and so on, we constructed the sidewalk detecting algorithm's as linear based. We get the linear features of the image that has more relibality to be a sidewalk and analysis the most data to get more accuracy and less elapsed time. We reduced it's time complexity by combining brief algorithm's result which operated in each video frame and analysis by some statistical regression method. The analysis' result are more accurated than just adopting single complicated fuction to each input video frame and more faster than just doing complicated algorithm to each video frames. So we made some brief algorithm that testify the line's reliablity and features to find whether the lines' are worth to regard as sidewalk or not. 
\newline 
Second is that the algorithm should work in real-time. The algorithm is designed for visually impaired person and used in working the sidewalks. In order to cope with each situation, it required that it's elapsed time be less than user's walking time and it be no supurious response and have high accuracy rating in finding sidewalk result and result's trend. In case of some suprusious response has occured, users only to believe program and follow the way what said to be result of algorithms. So if the result is not suited for actual sidewalk, it is going to be a big problem and the algorithms' performance will be discussed. Also, if sidewalk's tendency is trembled as every steps of user, it is hard to user to recoginze what is acutual path to follow, as there exist so many path as they walk. In short, it is neccsiated to reduce elapsed time as work in real-time and to improve algorithms accuracy to never to confuse user by it's result. 
\newline
So in order to achieve this two purpose, with the help of computer vision, we have developed a Walking Assistance Android Application(WAAA) for the visually impaired. In order to develop WAAA, therefore, we need algorithms that can detect sidewalks and extract the surrounding noise. At first, we planed to use color histogram analysis to detect sidewalk. We studied various algortihms concerning sidewalk detection, and we found a algorithm which based on color histogram analysis. And it works reliably on the ideal sidewalk situation, which there are only one pattern and color, and also no unexpected noise. But it turned out that, in reality, it didn’t work well. Because sidewalk can be covered partly with the shadow of trees, some kind of obstacles, other pedestrians or abnormal pattern and color. For this reason, we need to make another algorithm that doesn’t use color historgram as much as we can. Finally, we decided to make an algorithm based on line detection. Through this algorithm, we could solve limitations which are caused by color histogram.
\newline
In this paper, we focus on the line detection and tendency of realtime images. We have studied how visually impaired people are walking on the sidewalk and what can usually happen in real world. The real world means that there are many fallen leaves. There are some pedestrians and many colors of blocks. Besides, there might be some corners where people must turn the direction at the sidewalk. So, we made algortihms to solve these problems in line detection.
\newline
In our algorithms, by comparing the subsequent video frames with realtime video frames, we increase the reliability of direction. This algorithm has little to do with machine learning, but it looks for accumlated input data set and evaluates the actual direction of pedestrian wants just like machine learning does. The goal of this algorithm is to detect the direction of users and the direction that users should follow. It gives users the information on which ways are better to follow and how to do that. As this algorithm is applied to WAAA, visually impaired people can recognize the correct direction and can make a decision about his or her way. 
\section{Related works}
There are some algorithm that can detect lines in real-time. Such as sidewalk as we studied, automobile's lane. Algorithms for Drone, for automobile is actively discussed. Such as Y HE(Y He, Yinghua, Hong Wang, and Bo Zhang. "Color-based road detection in urban traffic scenes." Intelligent Transportation Systems, IEEE Transactions on 5.4 (2004): 309-318.)'s road following algorithms and so on. \newline
In previous algorithms, there are mostly depends on colors of images. A road following algorithm developed by Dahlkamp et al uses Gaussian color model. A benchmark road follower system, SCARF(Supervised Classification Applied to Road Following) also uses Gaussian color model too. However, There are many limitations in applying  algorithms in real world. So, we gave up using the road detection system using Color Histogram and utilized edge detection along with road shape model to detect road areas in an image. Adopting this system, we made a blueprint in our research. Contrary to their studies, we used line detection and linear regression to get reliabe line in ours and accmulated the tendency lines and found the direction of user. \newline
We first had to apply some filter to reduce noise of the image. Tomasi (Tomasi, Carlo, and Roberto Manduchi. "Bilateral filtering for gray and color images." Computer Vision, 1998. Sixth International Conference on. IEEE, 1998) has presented the Bilteral filter by way of reduce noise of the image. It is smoothing filter for images that is characterized by being non-linear, edge-preserving, and noise-reducing. G Deng (Deng, G., and L. W. Cahill. "An adaptive Gaussian filter for noise reduction and edge detection." Nuclear Science Symposium and Medical Imaging Conference, 1993., 1993 IEEE Conference Record.. IEEE, 1993.) has present an adaptive Gaussian filter for noise reduction and edge detection. In order to get more specific information of image, although Bilteral filter has more time complexity than Gaussian's, we adopted the previous one. Tomasi has present the way to use this filter in RGB or other color model images.\newline
After applying filters, we need to find countours of the image. To detect the image's contours, we choose Canny detection method by J. Canny(Canny, John. "A computational approach to edge detection." Pattern Analysis and Machine Intelligence, IEEE Transactions on 6 (1986): 679-698.). Canny detection has got a two threshold to find countours. By adjusting two threshold it find edges by Sobel edge detector. After that, it check ecah edge's magnitude is maximum. If it is maximum, it only apply it and connect with other edges. It has got low error ratio, no spurious response, and well localized algorithm. It's result has mainly affected by this parameter. But defining two threshold's value is not easy. So P Bao(Bao, Paul, Lei Zhang, and Xiaolin Wu. "Canny edge detection enhancement by scale multiplication." Pattern Analysis and Machine Intelligence, IEEE Transactions on 27.9 (2005): 1485-1490.) has suggested scale multiplication to enhance the quality of detection. In our algorithm, we resized the image frame and set value by some test.\newline
To adjust appropriate threshold, video frame's color is need to be equalized. Histogram Equalization(HE) mehtod is well known algorithm to revise image. Shah(Shah, Ghous Ali, et al. "A REVIEW ON IMAGE CONTRAST ENHANCEMENT TECHNIQUES USING HISTOGRAM EQUALIZATION." Science International 27.2 (2015)) arranged this method to use. We equalized the video frame before applying The filter. The image is well-ordered by resizing and equalizing method. There are some adaptive filter and method. But our work is real-time image processing and needed to be operated in less time complexity. So we just set the threshold as a constant value. \newline
In order to detect sidewalks, we have to find some lines in the image. There are many algorihtms that find images's line. But our algorithm is primarily based on the Probabilistic Hough Transform. Probabilistic Hough Transform(PHT) by Kiryati(Kiryati, Nahum, Yuval Eldar, and Alfred M. Bruckstein. "A probabilistic Hough transform." Pattern recognition 24.4 (1991): 303-316.) and Progressive Hough Transform(PPHT) by Barinova (Barinova, Olga, Victor Lempitsky, and Pushmeet Kholi. "On detection of multiple object instances using hough transforms." Pattern Analysis and Machine Intelligence, IEEE Transactions on 34.9 (2012): 1773-1784.) are good way to detcet line features in the image. It is a technique which can be used to isolate features of a particular shape within a image. By adopting this function to countours, we get line features of each image frame. 
Succeeding finding lines of image, we verified each line's reliablity by some equations. \newline
Finally, we get two most reliable lines on each video frame and accumulated it. Subsequently, we do linear regression on stacked lines. Seber(Seber, George AF, and Alan J. Lee. Linear regression analysis. Vol. 936. John Wiley \& Sons, 2012.) has arranged some linear regression method. Such as Ordinary Least Squares(OLS), Generalized Least Squares(GLS), Total Least Squares(TLS) and so on. Despite the fact that GLS and TLS is more reliable for analysing data, we chose OLS to get less elapsed time to be real-time processing. It is popular and powerful estimator and it's result line has to go through average of the points. We, so as to, adopted OLS estimator to regression to get better results. 
\section{Algorithms}
The main key of our algorithm is summarizing. We apply small patch and accumulated each result. Then we summarized the result and get results. It is quite effective than just do high-accuracy algorithm to each video's frame. We reduce much time by adopting this method to get data. It is economical for doing just little time-complexity procedure and summurizing. Thanks to linear regression analysis, we easily get total data from accumulated data and it's suit well for real-time processing for program lkie this.
\newline\newline 
Before we applied the algorithm, there were some steps to go through. We got the video frame by server. But, to use this video frame, the video frames must be converted to countours image. To minimize elapsed time, we applied Bilateral filter to exclude frames' unexpected noises. After applying filter, we applied Canny Detection to find countours of the images. Also, we used Probabilistic Hough Transform to get lines of the images. After that, we measured the relibality of the each line. And the most reliable two lines are results. But, two lines must be different from each other. So, we set equation of M like this.
\begin{equation}
\mathit{{M}_{ij} = {K}_{i} \times {m}_{j}} \quad\left(i=0,1 \right) \left(j=0,1 \right)
\end{equation}
Equation (1) simply represents the range of line's slope. M is the discriminant of the line that represent the upperbound and lowerbound of the should not be required image's slope. We set value like this. K is the weighted value and it would be defined as follow by given test.
\begin{equation}{K}_{0} = 0.9,\quad{K}_{1} = 1.1\end{equation} 
To set the slope's range, we set K as the constant value which make the candidate of slopes. As lines' slope is described as DMS notation width of range is defined by each line's slope. And there are more candidates of line where test line's slope is high. To get reasonable path, it is requrired that angle of slope is close to 90 degree. Lines are required to contrary to each other by it's direction and used to estimate the sidewalk's actual direction by finding vanishing point which made of each side's line that assumed to be sidewalk. 
\newline If the two lines’ slope is too close, it means there are no need to consider these lines because two lines are almost one line in high probability. In other case, if some new line has come in the situation of it's slope is not included in existing line's range, we define the situation as worthwhile and check line's relibality more. We check the line's relibality by To check lines’ reliabilty, we only compared a new line with two lines that already exist. Since we wanted to check all lines’ reliablity, we apply this equation in all each input line. \newline
In order to be reliable, it is required to be long and should be exclusive for each. Also, as we just need two line of result, the two result line's slope must be different. Now, we set new discriminant of slope like this. 
\begin{equation}
{N}_{ij} = {m}_{i} \times {p}_{j} \quad (i=0,1)(j=0,1)
\end{equation}
Discriminant (3) means range of slope that can admitted. Although the case that slope is not included in this range is better than the others, if the existing line's reliablity is hard to believe in only slope's value, We look for line's length also and set equation like this. The following situation is show as Figure 1. 
\begin{equation}
\mathit{S\left(l,m\right) = \sum_{i=0}^{n}{(\left(\beta{l}_{i}^{2} + D({m}_{i}\right) - (\beta{l}^{2} + D\left(m \right)))}^{2}}
\end{equation}
We get implict derivatives of this function by length to get minimum data of loss function's result. It's paramter are easily changed to improve its quailty, so we, at first, coordinate the parmater to suit with input data. Our purpose is to get minimum variance of line, to accomplish,  
\begin{equation}
\mathit{\frac{\partial S\left(l,m\right)}{\partial l} = \sum_{i=0}^{n}{\left(4{l}^{2}-4{{l}_{i}^{2}}\right)} = 0} 
\end{equation}
Firts, we get partial derivate of lines that can be drawn a conclusion of finding extreme value. As we found, in the equation, It is required that each sum of length's square is accrodance. Next, we get partial derviate of loss fuction by slope.
\begin{equation}
\mathit{\frac{\partial S\left(l,m\right)}{\partial m} = \sum_{i=0}^{n}{({D}^{\prime}(m)D(m) - \beta{l}^{2})} = 0}
\end{equation}
To arrange equation (5), equation (6), We convert them form that we can use. We extract this relations. Minimum price of loss function have to occured in such place that two variables are also reached to critical value.
\begin{equation}
\mathit{{D}^{\prime}(m)D(m) = \beta{l}^{2},\quad\sum_{i=0}^{n}{{l}_{i}^{2}} = n{l}^{2}}
\end{equation}
\begin{equation}
\mathit{D(m)=\pm\sqrt{2}\sqrt{c+\beta{l}^{2}m}}
\end{equation}
By computing data with this function, We get some data of parameter in equation. Figure 3 shows an data set of input and output to get parameters. We compare line's reliablity by S(l,m) and it's been requried to be good method. To simply the method, we define the parameter c as 0, so The equation is simply by
\begin{equation}
\mathit{D(m)=\pm l\sqrt{\beta m}}
\end{equation}
When calculating this function, l which shows line's size is constant value and only m is dependant variable. After showing the definiation of function D and loss function S, we can now set the actual fuction that evaluate whether line's reliabilty is more good than another or not. This method is simple by indicating loss function in one sum.
\begin{equation}
\mathit{T(l,m) = ((\beta l^{2}+D(m)) - (\beta {l}_{avg}^{2}+D(m_{avg}))^{2}}
\end{equation}	
\end{document}